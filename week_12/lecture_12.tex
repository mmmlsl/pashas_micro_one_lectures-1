\documentclass{beamer}
\usepackage[russian]{babel}
\usetheme{metropolis}

\usepackage{amsthm}
\usepackage{ulem}
\setbeamertemplate{theorems}[numbered]

\setbeamercolor{block title}{use=structure,fg=white,bg=gray!75!black}
\setbeamercolor{block body}{use=structure,fg=black,bg=gray!20!white}

\usepackage[T2A]{fontenc}
\usepackage[utf8]{inputenc}

\usepackage{hyphenat}
\usepackage{amsmath}
\usepackage{graphicx}

\AtBeginEnvironment{proof}{\renewcommand{\qedsymbol}{}}{}{}

\title{
Микроэкономика-I
}
\author{
Павел Андреянов, PhD
}

\begin{document}

\maketitle

\begin{frame}{План лекции}
\begin{itemize}
  \item Часть 1. Повторение РВ, производство, I и II теоремы благосостояния, доли фирм, равновесие Эрроу-Дебре.
  \item Часть 2. Избыточный спрос, существование и единственность равновесия.
\end{itemize}
\end{frame}

\section{Повторение РВ}
\begin{frame}{Повторение РВ}
Рассмотрим экономику с тремя агентами: $A,B,C$, и двумя товарами: $x, y$.
Пусть начальные запасы равны: (1,0), (0,2) и (1,1) соответственно, а суммарные запасы равны (2,3). Цены товаров назовем $p,q$.

Пусть полезности имеют вид Кобб Дугласа $\log x + \log y$, а множители лагранжа обозначим как $\lambda, \mu, \gamma$.

Выпишем систему уравнений для поиска равновесия. Неизвестные это
$$x_a, y_a, x_b, y_b, x_c, y_c, \lambda, \mu, \gamma, p,q$$
причем нормировка $q = 1$. То есть, 10 неизвестных.
\end{frame}

\begin{frame}{Повторение РВ}
Первый блок уравнений это сам ящик эджворта
\begin{gather*}
	x_a + x_b + x_c = 2\\
	y_a + y_b + y_c = 3
\end{gather*}
это 1,2 из 10 необходимых уравнений
\end{frame}

\begin{frame}{Повторение РВ}
Второй блок уравнений это УПП
\begin{gather*}
	1/x_a = \lambda p, \quad 1/y_a = \lambda q\\
	1/x_b = \mu p, \quad 1/y_b = \mu q\\
	1/x_c = \gamma p, \quad 1/y_c = \gamma q
\end{gather*}
это 3,4,5,6,7,8 из 10 необходимых уравнений.

\alert{Внимание:} Если бы полезность агента $B$ была Леонтьев, то вместо двух условий первого порядка было бы 1 условие <<вершины уголка>>, но не было бы множителя Лагранжа $\mu$. То есть, на 1 уравнение меньше и на 1 неизвестную меньше.

\end{frame}

\begin{frame}{Повторение РВ}
Третий блок уравнений это Законы Вальраса
\begin{gather*}
	p x_a + q y_a = 1 \cdot p + 0  \cdot  q\\
	p x_b + q y_b = 0 \cdot p + 2 \cdot q \\
	\alert{p x_c + q y_c = 1 \cdot p + 1  \cdot q}
\end{gather*}
это 9,10 из 10 необходимых уравнений.

\alert{Внимание:} Последний закон Вальраса не считается, так как он линейно зависим с остальными в ящике эджворта.
\end{frame}

\section{Экономика с производством}
\begin{frame}{Экономика с производством}
\alert{Экономика с производством} это

\begin{itemize}
  \item несколько потребителей (индивидуумов) $I = \{ a,b,c, \ldots\}$
  \item несколько производителей $J = \{ \alpha, \beta, \gamma, \ldots\}$
  \item несколько товаров $K = \{1, 2, 3, \ldots\}$
  \item начальные запасы $\vec w_{a}, \vec w_{b}, \vec w_{c}, \ldots$
\end{itemize}

В этой модели запасы есть только у потребителей. У производителей ничего нет, кроме технологий.

Технология производителя $j$ описывается <<классическим>> технологическим множеством $Y_j$, то есть, удовлетворяющим всем классическим аксиомам: рог изобилия, выпуклость, и.т.д...

\end{frame}

\begin{frame}{Экономика с производством}
Сразу определим вспомогательный объект.

\alert{Совместным технологическим множеством} $Y$ экономики с производством называется сумма всех индивидуальных технологических множеств $Y_j$.

Подсчет $Y$ дословно соответствует тому, что мы делали раньше, когда надо было "объединить два завода". 

Как правило, $Y$ выглядит примерно так же, как и сами $Y_j$: он выпуклый, проходит через ноль, и не содержит первый ортант.
\end{frame}

\begin{frame}{Экономика с производством}
\begin{center}
     \includegraphics[width=.8\textwidth]{./tech.png}
     \end{center}

\end{frame}

\begin{frame}{Экономика с производством}

\alert{Допустимым состоянием} экономики с производством называется набор координат потреблений и производств
$$\vec x = \{\vec x_i\}_{i \in I}, \quad \vec y = \{ \vec y_j\}_{j \in J}$$

такой, что ... все то же что и в экономике обмена ... плюс производства принадлежат соответствующим технологическим множествам $\vec y_j \in Y_j.$

Так, а причем тут совместное технологическое множество?

\end{frame}

\section{Ящик Эджворта}

\begin{frame}{Ящик Эджворта}

Как представить себе ящик Эджворта в экономике с производством?

\begin{itemize}
  \item сначала надо нарисовать суммарные запасы $\{w_k\}_{k \in K}$, другими словами, угол ящика <<до производства>>
  \item затем надо построить совместное технологическое множество $Y$, проходящее через $\{w_k\}_{k \in K}$, как если бы это было начало координат
  \item затем на (или даже под) границей этого множества выбрать новую точку - угол ящика <<после производства>>
  \item от нее нарисовать ящик для уже чистого обмена
\end{itemize}

Пространство допустимых состояний описывается множеством прямоугольников (ящиков) в $\mathbb{R}^2$, по одному на каждый $y \in Y$.

\end{frame}

\begin{frame}{Ящик Эджворта}
В проекции на пространство потреблений, ящик уже вовсе не ящик. Назовем его <<обобщенным ящиком>> Эджворта $\tilde E$. 
\begin{center}
     \includegraphics[width=.7\textwidth]{./tech2.png}
     \end{center}

\end{frame}

\begin{frame}{Ящик Эджворта}

Если производство эффективное, то получается что угол ящика Эджворта как бы <<едет вдоль границы>> $Y$. Как только мы зафиксировали производство $y \in Y$, ящик останавливается и дальше анализ соответствует простой экономике обмена.

Конечно, неэффективное производство тоже является допустимым, но мы никогда не будем по настоящему рассматривать его.

Забегая вперед, неэффективное производство не может быть Парето оптимальным с локально ненасыщаемыми полезностями.

\end{frame}

\section{Парето оптимальность}

\begin{frame}{Парето оптимальность}

Парето-оптимальность определяется аналогично тому, как мы это делали в экономике обмена, только поменялась интерпретация допустимого состояния.

Допустимое (в эк-ке с пр-вом) состояние $x,y$ \alert{слабый ПО}, если не существует другого допустимого состояния, которое делает всем агентам строго лучше.
$$ \tilde E \cap L^1_{++}(x) \cap L^2_{++}(x) = \emptyset, \quad y \in Y$$
Допустимое (в эк-ке с пр-вом) состояние $x,y$ \alert{сильный ПО}, если не существует другого допустимого состояния, которое делает всем агентам не хуже, но хотя бы одному агенту строго лучше.
$$ \tilde E \cap ((L^1_{++}(x) \cap L^2_{+}(x)) \cup (L^1_{+}(x) \cap L^2_{++}(x))) = \emptyset, \quad y \in Y$$

\end{frame}

\section{Геометрическая интерпретация}

\begin{frame}{Геометрическая интерпретация}

Назовем ПО \alert{внутренним} если он находится <<внутри>> ящика Эджворта. Я могу дать геометрическую интерпретацию внутреннего ПО, это точка, в которой:
\begin{itemize}
  \item равны наклоны всех кривых безразличия
  \item равны наклоны всех технологических границ
  \item и все они не только равны друг другу но также равны наклону совместной технологической границы
\end{itemize}

Другими словами:
$$ \nabla U_i = \nabla F_j = \vec p, \quad \forall i,j.$$ 

\end{frame}

\begin{frame}{Геометрическая интерпретация}
Получается, что Парето фронт выглядит примерно так
\begin{center}
     \includegraphics[width=.8\textwidth]{./tech3.png}
     \end{center}

\end{frame}

\begin{frame}{Геометрическая интерпретация}

Что это значит? 

Это значит что производство эффективно распределено между заводами а потребление эффективно распределено между потребителями. 

Более того, мы можем притвориться, что $F(\vec y)$ это как бы полезность еще одного агента, которая зафиксирована на некотором уровне. 

В таком случае действует стандартный подход с взвешенной полезностью.

\end{frame}

\begin{frame}{Геометрическая интерпретация}
Пусть есть две полезности $U_a(\vec x)$ и $U_b(\vec y + \vec w - \vec x)$ и
$$\vec y \in \partial Y \quad \Leftrightarrow \quad F(\vec y) = 0.$$
В таком случае взвешенная полезность это
$$ \lambda U_a(x) + \mu U_b(y + w-x) + F(y) \to \max_{x,y}, \quad \lambda, \mu \geqslant 0$$
Максимизируем взвешенную полезность, добавляем ограничение $F(\vec y) = 0$, получаем Парето фронт.

Условия первого порядка:
$$ \nabla U_i = \nabla F_j = \vec p, \quad \forall i,j.$$ 

\end{frame}

\section{Равновесие Вальраса}

\begin{frame}{Равновесие Вальраса}

\alert{Равновесием Вальраса} экономики с производством называется допустимое состояние $\vec x, \vec y$ и вектор цен $\vec p$, такие, что каждый агент достигает максимума полезности по бюджетному ограничению, с бюджетом равным доходу от продажи своих начальных запасов плюс трансферт, а производители максимизируют прибыль.
$$ \forall i \in I, \quad \vec x_i \in arg \max U_i(*) \quad s.t. \quad \vec p \cdot * \leqslant \vec p \cdot \vec w_i + T_i$$
$$ \forall j \in J, \quad \vec y_j \in arg \max \pi_j(*) \quad s.t. \quad * \in Y_j$$
где $\pi_j(*) = \vec p \cdot *$ это прибыль фирмы $j \in J$. 

\end{frame}

\begin{frame}{Равновесие Вальраса}
Также, я бы добавил в определение равновесие \alert{денежное равенство}, без него могут начаться проблемы с нормировкой: $$\sum_i T_i = \sum_j \pi_j.$$
Другими словами, фирмы должны, по какой то причине, отдать все свои деньги потребителям.
\end{frame}

\begin{frame}{Равновесие Вальраса}
Частный случай такой конструкции это когда агент $i$ владеет долей $\gamma_{ij}$ от фирмы $j$. Тогда он получает трансферт $\sum_j \gamma_{ij} \pi_j$. 

Заметим что
$$ \sum_i T_i = \sum_i (\sum_j \gamma_{ij} \pi_j) = \sum_j (\sum_i \gamma_{ij}) \pi_j = \sum_j \pi_j,$$
поскольку доли каждой фирмы суммируются в единичку. 

Это называется \alert{равновесием Эрроу Дебре}.
\end{frame}

\section{РВ как система уравнений}

\begin{frame}{РВ как система уравнений}

Попробуем снова сосчитать уравнения и неизвестные.

Неизвестные тоже можно разбить на группы

\begin{itemize}
  \item цены, их $K$ штук,
  \item собственно потребления, их $K \cdot I$ штук
  \item производства, их $K \cdot J$ штук
  \item множители Лагранжа, их $I$ штук
\end{itemize}

Однако, одна из цен нормируется к единичке, так что $$(I +J)\cdot K + K + I - 1$$ неизвестных.

\end{frame}

\begin{frame}{РВ как система уравнений}

Сосчитаем уравнения:

\begin{itemize}
  \item товарные равенства, $K$ штук
  \item условия оптимальности агентов, $K \cdot I$ штук
  \item условия оптимальности фирм, $K \cdot J$ штук
  \item законы Вальраса, $I$ штук
\end{itemize}

Причем последний закон Вальраса не считается, так что $$(I +J)\cdot K + K + I - 1$$ уравнений.

\end{frame}

\begin{frame}{РВ как система уравнений}

Вообще говоря, непонятно, сколько решений имеет такая система нелинейных уравнений и есть ли они вообще. 

После перерыва мы убедимся, что, как правило, решение есть, и иногда (при дополнительных сильных предпосылках) оно единственно.
\end{frame}

\section{Первая теорема благосостояния}

\begin{frame}{Первая теорема благосостояния}

Эту теорему можно сформулировать двумя способами.

\alert{Первая версия}: Пусть $(\vec x, \vec y, \vec p)$ - равновесие в экономике с производством, тогда $(\vec x, \vec y)$ - \alert{слабый} Парето-оптимум.

\alert{Вторая версия}: Пусть $(\vec x, \vec y, \vec p)$ - равновесие в экономике с производством, где все полезности \alert{локально ненасыщаемы}, тогда $(\vec x, \vec y)$ - \alert{сильный} Парето-оптимум.

\end{frame}

\begin{frame}{Первая теорема благосостояния}

\alert{Доказательство (первая версия)}: 

Мы уже доказали ее для случая без производства, от обратного, то есть, предположив другую точку в ящике Эджворта (я называл его $E$) которая лежит в пересечении всех $L^i_{++}(\vec x)$, и придя к противоречию с максимизацией полезности, которая есть часть равновесия Вальраса. 

Здесь ситуация аналогичная, ящик тут <<обобщенный>> $\tilde E$. 

Но это будет один и тот же ящик $\tilde E$ в определении и ПО и в доказательстве, так что сути это не меняет.

Конец.

\end{frame}

\begin{frame}{Первая теорема благосостояния}

\alert{Доказательство (вторая версия)}: 

От обратного, предположим, что есть слабое Парето-улучшение, несмотря на то, что а) агенты максимизируют полезности б) фирмы максимизируют прибыль.

Пусть допустимая точка $(\tilde x, \tilde y)$ - это кандидат на Парето-улучшение по сравнению с допустимой точкой $(\vec x, \vec y)$ который есть часть равновесия Вальраса $(\vec x,\vec y,\vec p)$. 

\end{frame}

\begin{frame}{Первая теорема благосостояния}

\textbf{Точка зрения потребителей}

Из локальной не-насыщаемости следует, что $\tilde x$ должна лежать вне бюджетного ограничения для того агента, который строго предпочитает $\tilde x$ к $\vec x$, в противном случае он бы никогда не купил $\vec x$ в равновесии,
$$ \exists i \in I \quad \vec p \cdot \tilde x_i > \vec p \cdot \vec x_i$$
А для всех остальных агентов либо вне либо в точности на бюджетной гипер-плоскости,
$$ \forall i \in I \quad \vec p \cdot \tilde x_i \geqslant \vec p \cdot \vec x_i.$$
Тогда верно что
$$ \vec p \cdot \sum_i \tilde x_i > \vec p \cdot \sum_i \vec x_i.$$
\end{frame}

\begin{frame}{Первая теорема благосостояния}

\textbf{Точка зрения производителей}

Поскольку все фирмы максимизируют прибыль,
$$ \forall j \in J \quad \vec p \cdot \tilde y_j \leqslant \vec p \cdot \vec y_j$$
а значит
$$ \vec p \cdot \sum_j \vec y_j \geqslant \vec p \cdot \sum_j \tilde y_j,$$
или еще по другому можно записать как
$$\vec p \cdot (\sum_i \vec w_i + \sum_j \vec y_j) \geqslant \vec p \cdot ( \sum_i \vec w_i + \sum_j \tilde y_j)$$
\end{frame}

\begin{frame}{Первая теорема благосостояния}

По цепочке у нас получается что
\begin{gather*}
\vec p \cdot \sum \tilde x_i > \vec p \cdot \sum \vec x_i = \\
= \text{товарооборот в равновесии } (\vec x,\vec y, \vec p) = \\
= \vec p \cdot (\sum \vec w_i + \sum \vec y_i) \geqslant \\
\geqslant \vec p \cdot (\sum \vec w_i + \sum \tilde y_i)
\end{gather*}

Однако, это противоречит тому, что $(\tilde x, \tilde y)$ это допустимое состояние, в которой $$ \forall k \in K: \quad \sum_{i \in I} \tilde x_{ik} = \sum_{i \in I} w_{ik} + \sum_{j \in J} \tilde y_{jk}.$$
Конец
\end{frame}

\section{Перерыв}

\section{Избыточный спрос}

\begin{frame}{Избыточный спрос}

\alert{Избыточным спросом} в экономике обмена/с производством/ЭД называется вектор $\vec E(\vec p) = \{E_k(\vec p)\}_{k=1}^n$, такой что:
$$
E_k(\vec p) = - \sum_{i \in I} w_{ik} - \sum_{j \in J} y_{jk} + \sum_{i \in I} x_{ik},
$$
где $\vec x, \vec y$ – это решения задач потребителя и производителя при данных ценах.

\end{frame}

\begin{frame}{Избыточный спрос}

Избыточный спрос обладает тремя свойствами:

\begin{itemize}
  \item $E(\vec p) = \vec 0$ в равновесии
  \item $p E(\vec p) = \vec 0$ всегда (если лок. ненасыщ)
  \item $E(\lambda \vec p) = E(\vec p)$ для всех $\lambda > 0$.
\end{itemize}

Поэтому, избыточный спрос, как правило, определен на симплексе цен - пространство цен после нормировки. Только нормировать я буду не как обычно (последняя цена к единице), а так, чтобы сумма цен равнялась единице.

\end{frame}

\begin{frame}{Избыточный спрос}

\alert{Симплексом цен} называется пространство цен $\Delta$ такое, что 
$$\Delta = \{p \in \mathbb{R}^K_{+}: \sum_{k \in K} p_k = 1 \}$$ 
Это, по сути, все цены кроме $\vec p = \vec 0$.

То есть, можно думать про равновесие Вальраса как про решение системы уравнений $E(\vec p) = \vec 0$ на симплексе $\Delta$. 

Заметим, что последнее уравнение в системе линейно зависимо, так что его нужно игнорировать.

\end{frame}

\section{Примеры}

\begin{frame}{Пример 1}

Рассмотрим два агента с начальными запасами (1,0) и (0,1) и полезностями вида Кобб-Дуглас. Производства нет.

\begin{itemize}
  \item выпишите избыточный спрос с ценами $\vec p = (p,1)$
  \item найдите равновесие
\end{itemize}

\end{frame}

\begin{frame}{Пример 2}

Рассмотрим два агента с начальными запасами (1,0) и (0,1) и полезностями вида Кобб-Дуглас, и одного агента с начальным запасом (1,1) и полезностью Леонтьев. Производства нет.

\begin{itemize}
  \item выпишите избыточный спрос с ценами $\vec p = (p,1)$
  \item найдите равновесие
\end{itemize}

\end{frame}

\begin{frame}{Пример 3}

Рассмотрим 1 агента с начальным запасом (1/2,0) и полезностью вида Кобб-Дуглас, и одну фирму (которой он же и владеет) с технологией $x - 2y \leqslant 0$
\begin{itemize}
  \item выпишите избыточный спрос с ценами $\vec p = (p,1)$
\end{itemize}

\end{frame}

\begin{frame}{Пример 4}

Рассмотрим одного агента с начальным запасом (1,0) и линейной полезностью, и другого агента с начальным запасом (0,1), полезностью Кобб-Дуглас и фирмой (которой он владеет) с технологией $\log(1-x) + 2\log(1-y) \geqslant 3 \log 2$
\begin{itemize}
  \item выпишите избыточный спрос с ценами $\vec p = (p,1)$
\end{itemize}

\end{frame}

\begin{frame}{Пример 5}

Рассмотрим одного агента с начальным запасом (1,0), полезностью Кобб-Дуглас и двумя фирмами (которыми он владеет) с технологиями $\log(1-x) + 2\log(1-y) \geqslant 3 \log 2$ и $2 \log(1-x) + \log(1-y) \geqslant 3 \log 2$
\begin{itemize}
  \item выпишите избыточный спрос с ценами $\vec p = (p,1)$
\end{itemize}

\end{frame}

\section{Существование решения}

\begin{frame}{Существование решения}

Предлагается следующая динамика цен на внутренности $\Delta$.

\begin{itemize}
  \item инициализируем точку $\vec p_0$
  \item пусть мы находимся в точке $\vec p_{i}$
  \item меняем цены по следующему алгоритму
$$ \tilde p_{k, i+1} := \vec p_{k, i} + \gamma \cdot \max(0, E_k(\vec p)), \quad \gamma >0$$
  \item и возвращаем их на симплекс
$$ \vec p_{k, i+1} := \frac{\tilde p_{k, i+1}}{\sum_{k \in K} \tilde p_{k, i+1}}.$$
\end{itemize}

Заметим, что РВ характеризуется стационарной точкой.

\end{frame}

\begin{frame}{Существование решения}

Это почти как градиентный спуск, но не совсем.

\begin{center}
     \includegraphics[width=.8\textwidth]{./simplex.png}
     \end{center}

\end{frame}

\begin{frame}{Существование решения}

Нас интересует фиксированная точка следующего отображения:
$$ \forall k, \ p_k \to \frac{p_k + \gamma \cdot \max(0, E_k(\vec p))}{\sum_j p_j + \gamma \cdot \sum_j \max(0, E_j(\vec p))}, \quad \gamma > 0,$$
которое действует из внутренности симплекса $\Delta$ на себя.

Заметим, что $\sum_j p_j = 1$ так как мы стартовали из симплекса.

По \alert{Теореме Брауэра}, непрерывное отображение компактного выпуклого множества на себя имеет неподвижную точку.

Осталось доказать, что она то и будет равновесием.

\end{frame}

\begin{frame}{Существование решения}

Пусть для некоторого вектора цен
$$ p_k = \frac{p_k + \gamma \cdot \max(0, E_k(\vec p))}{1 + \gamma \cdot \sum_j \max(0, E_j(\vec p)))}$$
тогда
$$ p_k + p_k \cdot \gamma \cdot \sum_j \max(0, E_j(\vec p))) = p_k + \gamma \cdot \max(0, E_k(\vec p))$$
или 
$$ p_k \cdot \sum_j \max(0, E_j(\vec p))) =  \max(0, E_k(\vec p))$$

\end{frame}

\begin{frame}{Существование решения}
напомним что
$$ p_k \cdot \sum_j \max(0, E_j(\vec p))) =  \max(0, E_k(\vec p))$$
умножим все на $E_k(\vec p)$...
$$ E_k(\vec p) \cdot p_k \cdot \sum_j \max(0, E_j(\vec p))) =  E_k(\vec p) \cdot \max(0, E_k(\vec p))$$
и сложим
$$ (\sum_k E_k(\vec p) \cdot p_k) \cdot \sum_j \max(0, E_j(\vec p))) =  \sum_k E_k(\vec p) \cdot \max(0, E_k(\vec p))$$
левая часть обнулится по закону Вальраса ($p E(p) = 0$)
\end{frame}

\begin{frame}{Существование решения}

получается 
$$\sum_k \max(0, (E_k(\vec p))^2) = 0 \quad \Rightarrow E_k(\vec p) = 0, \forall k$$
Осталось вспомнить, какие условия необходимы для непрерывности спросов и выполнения Закона Вальраса
\begin{itemize}
  \item непрерывность всего
  \item выпуклость всего
  \item локальная ненасыщаемость предпочтений
\end{itemize}

Конец доказательства?
\end{frame}

\begin{frame}{Существование решения}
На самом деле нет. 

Отображение должно действовать из всего компакта $\Delta$, а у нас границе $\Delta$ бесконечные избыточные спросы. 

Это легко исправляется переопределением избыточного спроса так, чтобы он никогда не превышал максимальное число товара в экономике (с учетом производства), то есть, размеры <<обобщенного ящика>> Эджворта.
$$
E_k(\vec p) := \min(E_k(\vec p), \text{ size of Edgeworth box}).
$$
Действительно, такая модификация не изменит определение РВ, но оно корректно определит динамику на границе симплекса. Теперь конец.

\end{frame}

\section{Единственность решения}

\begin{frame}{Единственность решения}
Вообще говоря, единственность неочевидна
\begin{center}
     \includegraphics[width=.8\textwidth]{./three.png}
     \end{center}
\end{frame}

\section{Единственность решения}

\begin{frame}{Единственность решения}
Однако есть не очень сильные предположения, при которых ее можно все-таки доказать.

Скажем, что избыточный спрос $\vec E(*)$ обладает свойством \alert{валовых субститутов}, если для двух векторов цен $\vec p, \vec q,$ \\таких что 
\begin{itemize}
  \item для одной координаты $p_i = q_i$
  \item для другой координаты $p_j > q_j, \ j \neq i$
  \item для остальныx координат $p_k \geqslant q_k, \ k \neq i, \ k \neq j$
\end{itemize}
верно что $E_i(\vec p) > E_i(\vec q)$. 

Понятно, что ВС избыточного спроса вытекает из ВС индивидуальных спросов и предложений каждого агента.

\end{frame}

\begin{frame}{Единственность решения}

Предположим, от обратного, что есть два вектора цен $\vec p \neq \vec q$, и вокруг каждого из них можно построить равновесие.

Рассмотрим число $\lambda$ такое, что:
$$ \lambda = \max_k\ (\frac{p_k}{q_k}),$$
\pause тогда существует индекс $i$ такой что
$$ q_i \cdot \lambda = p_i,$$
\pause существует индекс $j$ такой что
$$ q_j \cdot \lambda > p_j,$$
\pause а для всех остальных индексов
$$ q_k \cdot \lambda \geqslant p_k.$$
\end{frame}

\begin{frame}{Единственность решения}

Тогда рассмотрим вектора цен $\vec p$ и $\lambda \vec q$

Напомним что существует индекс $i$ такой что
$$ q_i \cdot \lambda = p_i,$$
\pause существует индекс $j$ такой что
$$ q_j \cdot \lambda > p_j,$$
\pause а для всех остальных индексов
$$ q_k \cdot \lambda \geqslant p_k.$$

\pause Другими словами, это предпосылка ВС, следовательно $$ E_i (\lambda \vec q) > E_i(\vec p).$$
\end{frame}

\begin{frame}{Единственность решения}

Наконец, надо вспомнить свойство гомогенности спроса
$$ E(\vec q) = E(\lambda \vec q) \neq E(\vec p)$$
Что противоречит тому, что $\vec p, \vec q$ были оба равновесиями.

\end{frame}

\end{document}
